\newpage


\centerline{\fangsong\bf\zihao{-2}{基于点云数据的雀替轮廓线参数信息自动化提取方法}}
\addcontentsline{toc}{section}{摘要(关键词)}%加入目录


\vskip 1cm

\begin{center}
	\kaishu
	\hspace{2cm}建筑与城市规划学院地理空间信息工程 \quad 张三 
	\vspace{5bp}
	\newline
	学号:2023111111
\end{center}

\vskip 10bp

{
\kaishu	
\hspace{5bp}{\zihao{-4}\textbf{【摘要】}} 
近年来,古建筑遗产的科学保护与发展成为了热点和难点问题。雀替作为中国古代建筑中最常见
的大木构件,具有重要的结构功能和艺术价值,其形态相对特殊但具有一定的规律性。针对外轮廓具有明
显分段曲线特征的蝉肚形雀替,发展完善了一套基于点云数据的雀替外轮廓线参数信息自动化提取方法,
可服务于雀替信息的数字化留取。该方法由点云切片算法、几何特征点识别算法和曲线拟合算法组成,通
过对3个典型雀替的数字化信息提取,建议了该方法的关键参数取值,其中关键参数窗口边长可取为构件长
度的1/55,曲线拟合点个数可取为3。基于该方法可准确高效地提取雀替外轮廓线参数信息,为古建筑中的
雀替保护提供了重要参考。

\vskip 10bp

\hspace{5bp} {\zihao{-4}\textbf{【 关键词】}} 
点云数据;轮廓线参数信息;自动化提取方法;雀替
}