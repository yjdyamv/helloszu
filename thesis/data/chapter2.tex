\section{轮廓线参数信息提取}
三维激光扫描能够以非接触的测量方式实时快速地获取目标高精度的三维坐标,由于点云数据具有采样密度大的特性,数据量过于庞大,难以直接进行构件几何特征参数信息提取,且雀替这类构件形态较为复杂,目前针对此类构件特征自动提取的相关研究仍相对较少,在精度和效率方面有待发展。古建筑不同构件外部轮廓表达所需参数存在显著差异,雀替形态随历史演变分类较多,本文研究对象为蝉肚形雀替,该类雀替外部轮廓主要为直线或多段规则曲线,具有明显的分段曲线特性。该类构件外轮廓的几何特征参数可通过直线长度和曲线多项式系数进行表达。针对该类雀替的分段曲线特性,本研究拟采用分段曲线表达的方法构建雀替模型。其关键难点在于:①外轮廓边界线的提取;②边界线上多段曲线间拐点的识别;③多段曲线的拟合。针对这些难点,本文发展完善了一套基于点云数据的构件参数信息自动化提取方法,可更加准确高效地提取该构件的几何参数信息。整体思路如图1所示。

\subsection{点云切片}
古建筑构件中部分构件表面带有复杂浮雕纹理信息,在外部轮廓参数提取过程中易造成干扰,影响特征参数提取的精度和效率。针对此种情况,本文采用万程辉提出的一种基于投影面的点云切片算法,根据构件的不同特征进行点云切片处理,将点云对象精简为片状点云,可通过提取某一单层切片点云的特征点,减少不必要信息的干扰,提高后续曲线拟合点提取的精度与效率。在点云数据进行切片处理前,需将点云数据进行预处理以及确认分层轴线方向应和构件对称轴是否保持一致,若不一致需对模型进行坐标变换,将点云模型变换至中心线平行于z轴正方向且底面平行于xoy平面的方向上,获得可适用于切片的构件点云模型。该算法主要通过提取一定厚度的点云,将其投影到中心面上实现对点云数据的切片处理。首先以点云模型中某一点建立局部坐标系,确定模型切片的方向(本文以x轴正方向为例),遍历点云数据获得模型在x轴方向上的最大值xmax和最小值xmin,根据应用所需将点云在x轴方向上划分为m个等间距切片,切片厚度h,x=xmax-mxmin,生成m组与x轴方向相同的投影平面,将点云数据投影到相应的参考平面上,生成外轮廓形状切片点云,如图2所示,其中切片厚度h与点云数据的分布密度有关。切片厚度的设置应在允许误差范围以内,由于一般情况下点云分辨率为2 mm,本文在此建议切片厚度取值为2 mm。

\subsection{几何特征点提取}

基于点云切片算法获得的片状点云数据量仍相对较大,且可表征构件参数信息的几何特征点一般存在于边缘拐点等关键部位,因此需要进一步精简片状点云模型获得几何特征点,为提取构件参数信息奠定基础。本文采用一种综合了Harris二维图像检测角点原理[8-10]以及三维点云密度特征的三维几何特征点提取算法。这主要是借用Harris算法在二维图像上通过灰度变化检测角点的思想,将这种思想拓展到三维点云中,利用三维点云的密度特征,通过检测窗口内点云密度变化确定拐点。该算法很大程度上避免了对边缘提取的依赖
性,降低了提取难度和计算量,具有旋转不变性以及较
强的鲁棒性。

该特征点提取算法具体流程如下:①计算点云数据中单点的法向量,在该点上建立局部坐标系,z方向是法线方向;②以该点为中心建立边长为r的窗口,计算分析点云密度变化,如图3所示,若窗口内点云密度在某一个方向上发生变化,该点为边缘点,若两个方向密度均大幅度改变则认为是拐点;③遍历所有点云数据,提取点云中所有拐点。其中,合理的窗口边长r取值对于拐点的提取至关重要,参数r设定过大会丢失拐点信息,设定过小则会导致提取出伪拐点。本文将在第2节实例研究中对该参数的取值展开相关研究,建议了较为合理的参数值。


\subsection{曲线拟合}
本文采用最小二乘原理用解析表达式来逼近离散数据点,该方法曲线拟合的思想是将数据点相对于最终曲线的偏差平方和最小化来拟合最终的曲线,采用多项式方程来求解目标曲线,如图4所示。在曲线拟合过程中,逐步提高多项式的阶次进行拟合,当相关性系数首次达到0.95时,认为满足精度要求。