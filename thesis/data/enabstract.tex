\newpage
%%英文标题此处过长,分为两行排版
\centerline{\fangsong\bf\zihao{-2}{An Automatic Extraction Method for Parameter Information of }}

\vspace{10bp}

\centerline{\fangsong\bf\zihao{-2}{Sparrow Brace's Contour Line Based on Point Cloud Data}}

\addcontentsline{toc}{section}{Abstract(Key words)}

\vskip 20bp

\hspace{4bp} {\zihao{-4}\textbf{【 Abstract】}} 
In recent years, the scientific protection and development of ancient architectural heritage has become a popular and critical issue. As the most common large wooden component in ancient Chinese architecture, sparrow brace has important structural functions as well as artistic value, and its form is relatively special but has certain regularity. In this paper, based on the significant segmentation curve characteristics of cicada belly sparrow brace, an automatic parameter extraction method for the external contour line using point cloud data is developed, which can be use to digitize the information of sparrow brace. This method consists of point cloud slicing algorithm, salient geometric features point detection algorithm and curve fitting algorithm. By extracting the information of three typical sparrow braces, the key parameters of the method are recommended. Specifically, the window length recommended as 1/55 of length of the component and the number of curve fitting points can be taken as 3, respectively. Based on this method, the parameter of the contour line of the sparrow brace can be extracted accurately and efficiently, which provides an important reference for the protection of sparrow brace in ancient buildings.

\vskip 10bp

\hspace{5bp}{\zihao{-4}\textbf{【 Keywords】}}
point cloud data; parameter information of contour line; automated extraction method; sparrow brace


\vskip 20bp

\begin{flushright}
	\kaishu 指导教师:\ 你的老师 \hspace{3cm}{ }
\end{flushright}

\label{lastpage}%%%%显示总页数