\section{引言}

\subsection{研究背景及意义}

随着我国经济的持续快速发展,古建筑的有效保护在世界范围内日益受到关注。我国现存古建筑数量庞大,其中大部分为木质结构且安全性能面临严峻威胁。雀替作为中国古代建筑中最常见的大木构件,无论是前期安置于梁或阑额与柱交界之处以承托梁枋,还是经过历史演变被装饰化,在古建筑中都具有十分重要的作用。如何高效准确地表达该构件的形态现状,获取可靠的几何特征参数信息,进而实现该构件的信息留取与管理,是后期古建筑内部结构表达与日常运营维护的重要基础。
古建筑中雀替形式不同于传统构件,其形态相对特殊且较为复杂,图纸等数据获取途径匮乏且信息不完整,难以准确地记录雀替的历史演变过程以及形态现状。基于三维激光扫描的精细测绘技术为这一问题的解决提供了科学手段,该技术以点云数据的形式获取目标物体的高精度阵列式空间信息,表达出物体的现实形态特征。其中存在的问题是,采用该技术获得的点云数据体量庞大且分布不均匀,能够准确表达该构件的关键信息数量远少于点云数据量。该类构件的外部轮廓具有一定的规律性,对此可采用几何特征参数对其关键信息进行表达。目前针对此类构件几何特征参数提取的研究还相对较少,基于海量精细点云数据,自动高效地提取该构件的几何特征参数,是将三维激光扫描技术应用于古建筑遗产保护的关键问题。面向古建筑的现状留取与保护迫切需求,针对如何可靠高效地提取雀替类构件参数信息这一关键问题,本文以三维激光扫描技术获得的高精度点云数据为基础,完善了一套基于点云数据的雀替几何特征参数信息自动化提取方法,通过对典型雀替外轮廓参数提取,建议了相关参数取值,为后期古建筑保护提供科学依据。\cite{TDBZ202407002}

\subsection{本文主要工作}




\subsection{论文组织结构}

 本论文共分为七章,内容如下: 

 第一章为引言, 主要介绍了本论文的研究背景、意义, 主要工作及论文的组织结构.
 
 第二章为推荐系统概览,并分类介绍了包括了基于内容、基于系统过滤与混合型推荐算法的一些典型的推荐学习算法。
 
 第三章为预备工作,首先简要回顾了Bayesian Personalized Ranking(BPR)推荐算法, 并对其局限性进行了一些探讨。
 
 第四章为适应性采样策略,主要研究了通过融合内容信息提出了适应性采样策略改进已有的均匀采样策略。
 
 第五章为整体的算法框架, 将适应性采样策略融入已有的BPR推荐模型。
 
 第六章为实验论证,主要内容为在适应性采样策略下的推荐算法的实验表现。
 
 第七章为结论与展望,首先简要总结了本文的一些工作,并对接下来进一步的研究工作做了展望。
