\section{结论与展望}

\subsection{本文的主要内容}

本文针对外轮廓具有明显分段曲线特征的蝉肚形雀替,结合海量高保真点云数据,发展完善了一套基于点云数据的雀替参数信息自动化提取方法,服务于雀替信息的数字化留取。通过对3个典型雀替的数字化信息提取,建议了该方法的关键参数取值,其中关键参数窗口边长可取为构件长度的1/55,曲线拟合点个数可取为3,采用该方法能够可靠高效地提取该类雀替轮廓线参数信息,为后期古建筑保护提供科学依据。

\subsection{进一步的研究工作}

尽管本文实验证明通过加入内容信息的确有助于提高推荐效果,但是对于加入内容信息的适应性采样策略在整个学习过程每个阶段的影响仍然有待研究。同时对于一些已有的一些融合内容信息的推荐方法,比如采用Word2Vec技术,还需进一步的研究调查在这些融合内容信息的不同推荐方法中的特点,适用性及其局限性。