\section{实验论证}
如上文所述,在几何特征点提取过程中窗口边长
r是关键参数。为了有效地估计出参数的合理取值,本通过设计多次实验,将参数r选取不同参数值得到的结果对比分析:通过大量的参数分析,对于不同的雀替构件,本研究均明确了可较准确获得拐点的r取值范围。对于尺寸分别为499 mm×37 mm×253 mm、701 mm×49 mm×293 mm和447 mm×49 mm×249 mm的雀替构件,其r取值范围分别为7~9 mm、13~15 mm和7~10 mm。值得注意的是,r的取值与构件的尺寸存在一定的关联,由于多段曲线一般位于长度方向,本文在此建议将r值取为雀替构件长度方向尺寸的某一相对值,具体而言将r取为长度的1/55。对于3个雀替构件,r分别近似为9 mm、13 mm和8 mm,均在参数分析的合理范围内。各案例将r取为上限值、本文建议值和下限值的拐点提取效果见表2\cite{1012391475.nh,WOS:000979996300004}

\subsection{数据集}
好
\subsection{评测标准}

r~0.95

\subsection{实验过程与分析}
